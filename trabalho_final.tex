Aplicações e Problemas Computacionais em Matemática Discreta.
João Victor Mendes Freire e Guilherme Locca Salomão
5 de julho de 2019

Introdução
    O que é Matemática Discreta

        A Matemática Discreta é um ramo da matemática que não possui uma definição formal precisa e exata. Ela é descrita como o ramo que estuda conjuntos contáveis, ou seja, conjuntos finitos ou que possuem a mesma cardinalidade do conjunto dos naturais.
        Ao se observar o conjunto dos números reais, por exemplo, podemos escolher dois números arbitrários, como 0 e 1. Entre esses dois números, existe uma infinidade de outros números, e, portando, não podemos considerar 0 o “primeiro" e 1 o “segundo”. O 0,5 existe entre eles. Bem como o 0,05, e o 0,005. Logo, não conseguimos mapear a cardinalidade dos números naturais para cada número real.
        O conjunto dos inteiros é infinito assim como os reais, mas, diferentemente deles, é enumerável. Portanto, podemos mapear a cardinalidade dos naturais neles. 
        Assim sendo, embora contenha infinitos elementos, o conjunto dos inteiros é uma estrutura discreta, de interesse da matemática discreta, ao contrario dos reais, que é uma estrutura contínua.


    Por que é importante?
        Dispositivos digitais estão presentes em inúmeros contextos no mundo contemporâneo. Smartphones, computadores pessoais, servidores, semáforos, sistemas de um banco, os sistemas de um veículo e milhares de outros sistemas são digitais. Devido a isto, são construídos por circuitos lógicos, que funcionam baseados em estruturas discretas.
        A comunicação entre bancos, governos e até mesmo entre pessoas é segura quando se utiliza Criptografia para proteger os dados. A matemática discreta é bastanten importante em se tratando de desenvolver novos e melhores algortimos para criptografar e descriptografar dados privados.
        Além desse aspecto computacional, a matemática discreta oferece as ferramentas para que, por exemplo, um carteiro descubra qual a rota de entregas mais eficiênte.
        Estes são apenas dois exemplos de impactos da matemática discreta na vida de uma pessoa comum. Existem inúmeros outros, e ainda mais para aqueles interessados em aprofundar seus conhecimentos na área de Ciência da Computação.


    Tópicos de Interesse
        Alguns dos principais tópicos em matemática discreta são: teoria dos números, conjuntos, funções, relações, recorrências e teoria dos grafos.
        % detalhar e refinar mais essa parte

    Materiais e Métodos

Capitulo I
    Metodologia
    Resultados e discussões
    Considerações finais

Capitulo II
    Metodologia
    Resultados e discussões
    Considerações finais

Capitulo III
    Metodologia
    Resultados e discussões
    Considerações finais

Capitulo IV
    Metodologia
    Resultados e discussões
    Considerações finais

Capitulo V
    Metodologia
    Resultados e discussões
    Considerações finais

Referências
    Discrete Mathematics - Wikipédia, consulta em 30 de junho de 2019